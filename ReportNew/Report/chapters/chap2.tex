%%%%%%%%%%%%%%%%%%%%%%%%%%%%%%%%%%%%%%%%%%%%%%%%%%%%
% This will help you in writing your homebook
% Remember that the character % is a comment in latex
%
% chapter 2
\chapter{Verification and Physical level}
\label{chap2}

\section{Simulations}

To ensure the functionality of our project, we run some simulations on Modelsim, using some .asm scripts.

%\begin{figure}[ht]
%\centering
%\includegraphics[]{chapters/figures/wave1} 
%\caption{Waves from simulation}
%\label{fig:wave1}  % here is the figure label
%\end{figure}


\section{Synthesis}

After veryfing that our DLX works as expected, we synthesize it using a script(given in appendix ?).
The result is in the following figure:

%\begin{figure}[ht]
%\centering
%\includegraphics[]{chapters/figures/synthesized} 
%\caption{RTL of our DLX}
%\label{fig:synthesized}  % here is the figure label
%\end{figure}

After a first synthesis without constraints we obatin:

\begin{itemize}
\item f$_{CLK}$ = ;\\
\item Data arrival time = ;\\
\item Combinational area = ;\\
\item Non combinational area = ;\\
\item Total cell area = ;\\
\item Cell Internal Power = ;\\
\item Net Switching Power = ;\\
\item Total Dynamic Power = ;\\
\item Cell Leakage Power = ;\\
\end{itemize}


While, applying contraints: 

\begin{itemize}
\item f$_{CLK}$ = ;\\
\item Data arrival time = ;\\
\item Combinational area = ;\\
\item Non combinational area = ;\\
\item Total cell area = ;\\
\item Cell Internal Power = ;\\
\item Net Switching Power = ;\\
\item Total Dynamic Power = ;\\
\item Cell Leakage Power = ;\\
\end{itemize}

\section{Layout}

